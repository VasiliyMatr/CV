%%%%%%%%%%%%%%%%%%%%%%%%%%%%%%%%%%%%%%%%%

%
% Important note:
% This template requires the resume.cls file to be in the same directory as the
% .tex file. The resume.cls file provides the resume style used for structuring the
% document.

%
% Creator Peilin Li
% Contact me via twitter/wechat: @pe1l1nl1
% linkedin.com/peill and/or github/ppeill
% Inspired by Peppa Pig 
%%%%%%%%%%%%%%%%%%%%%%%%%%%%%%%%%%%%%%%%

%----------------------------------------------------------------------------------------
%	PACKAGES AND OTHER DOCUMENT CONFIGURATIONS
%----------------------------------------------------------------------------------------

\documentclass{resume} % Use the custom resume.cls style

\usepackage[left=0.40in,top=0.3in,right=0.75in,bottom=0.1in]{geometry} % Document margins
\usepackage{fontawesome}
\usepackage{times}
\usepackage{makecell}

\newcommand{\tab}[1]{\hspace{.2667\textwidth}\rlap{#1}}
\newcommand{\itab}[1]{\hspace{0em}\rlap{#1}}

\name{Vasilii Matrenin}

\address{\faGithub{github.com/VasiliyMatr} $\,$ \faEnvelope{vasiliymatrenin@gmail.com}}
\address{\faMapMarker{Russia, Moscow}}

\begin{document}
{\centerline {\em \textbf{\Large System Programmer}}}

%----------------------------------------------------------------------------------------
%   EDUCATION
%----------------------------------------------------------------------------------------
\begin{rSection}{Education}

\textbf{Moscow Institute of Physics and Technology} \hfill {\em 2020 - 2024} \\
\textit{Applied Physics and Mathematics, DREC Department, Bachelors degree} \\
\textit{Thesis: Synthetic execution traces generation for testing:}
    $\href{https://github.com/VasiliyMatr/bachelors_diploma}{\texttt{VasiliyMatr/bachelors\_diploma}}$

\textbf{Moscow Institute of Physics and Technology} \hfill {\em 2024 - Present} \\
\textit{Applied Physics and Mathematics, DREC Department, Masters student}

\end{rSection}

%----------------------------------------------------------------------------------------
%   SKILLS
%----------------------------------------------------------------------------------------
\begin{rSection}{Skills}

\begin{tabular}{ @{} >{\bfseries}l @{\hspace{6ex}}l }
Languages: & C/C++, Rust, Lua, Bash, Assembly, Python \\
\\
Software \& Tools: & LLVM, Sol2, GTest, CMake, Make, VHDL, Git, Linux, debugging, profiling \\
\\
Math: & Math Analysis, Linear Algebra, Diff Equations, Integrals, Digital Signals Processing \\
\\
Other Skills: & CPU micro-arch, CPU functional sim, PLs VMs, PLs Design, \LaTeX, English (C1)
\end{tabular}

\end{rSection}

%--------------------------------------------------------------------------------
%    PROJECTS
%--------------------------------------------------------------------------------
\begin{rSection}{Projects (All available on Github)}

\textbf{Functional RISC-V simulator} \hfill {\em Sep 2023 - Jul 2024} \\
Repository: $\href{https://github.com/VasiliyMatr/sim2023}{\texttt{VasiliyMatr/sim2023}}$ \\
Features: MMU, Plugins support, Threaded code

\textbf{LLVM practice} \hfill {\em Sep 2023 - Jul 2024} \\
Repositories: $\href{https://github.com/VasiliyMatr/LLVM_Practice}{\texttt{VasiliyMatr/LLVM\_Practice}}$,
    $\href{https://github.com/VasiliyMatr/llvm-project/tree/users/VasiliyMatr/bachelors-practice}{\texttt{VasiliyMatr/llvm-project}}$ \\
Features: Tracing instrumentation pass, Frontend for toy language (based on Flex + Bison), Backend for toy ISA

\textbf{JIT and AOT practice} \hfill {\em Sep 2024 - Jul 2025} \\
Repository: $\href{https://github.com/VasiliyMatr/jit_and_aot}{\texttt{VasiliyMatr/jit\_and\_aot}}$ \\
Features: IR Library, Graphs Alogos (DFS, RPO, Dom Tree, Loop Tree), Peepholes, Checks Elimination \\

\end{rSection}

%----------------------------------------------------------------------------------------
%   WORK EXPERIENCE
%----------------------------------------------------------------------------------------
\begin{rSection}{Work Experience}

\textbf{Huawei corporation, R\&D in CPU \& Software co-design} \hfill {\em Jul 2022 - Present}
\begin{itemize}
\item Implemented data processing/generation tools
\item Participated in future directions discussions
\item Managed small team (3 people)
\item Mentored 2 junior devs
\end{itemize}

\textbf{Intel corporation, IPP Domain internship} \hfill {\em Jul 2021 - Aug 2021}
\begin{itemize}
\item Investigated possibilities for Intel IPP data compression algos integration
\item Optimized Intel IPP data compression algos
\item My manager and mentor highly valued my work and sent good
    $\href{https://github.com/VasiliyMatr/CV/tree/master/PDF/ISI2021_feedback.pdf}{feedback}$
\end{itemize}

\end{rSection}

\end{document}