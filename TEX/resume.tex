%%%%%%%%%%%%%%%%%%%%%%%%%%%%%%%%%%%%%%%%%

%
% Important note:
% This template requires the resume.cls file to be in the same directory as the
% .tex file. The resume.cls file provides the resume style used for structuring the
% document.

%
% Creator Peilin Li
% Contact me via twitter/wechat: @pe1l1nl1
% linkedin.com/peill and/or github/ppeill
% Inspired by Peppa Pig 
%%%%%%%%%%%%%%%%%%%%%%%%%%%%%%%%%%%%%%%%

%----------------------------------------------------------------------------------------
%	PACKAGES AND OTHER DOCUMENT CONFIGURATIONS
%----------------------------------------------------------------------------------------

\documentclass{resume} % Use the custom resume.cls style

\usepackage[left=0.40in,top=0.3in,right=0.75in,bottom=0.1in]{geometry} % Document margins
\usepackage{fontawesome}
\usepackage{times}
\newcommand{\tab}[1]{\hspace{.2667\textwidth}\rlap{#1}}
\newcommand{\itab}[1]{\hspace{0em}\rlap{#1}}

\name{Vasilii Matrenin} % Your name

\address{\faGithub{ github.com/VasiliyMatr} $\,$ \faEnvelope{ matrenin.vn@phystech.edu}}
\address{\faMapMarker{ Russia, Moscow}} % Your address

\begin{document}
{\centerline {\em \textbf { C/C++ Developer } } }

%----------------------------------------------------------------------------------------
%	EDUCATION
%----------------------------------------------------------------------------------------
\begin{rSection}{Education}

{\bf Moscow Institute of Physics and Technology } \hfill {\em Sep 2020 - Jun 2022} 
\\{ \textit { Applied Physics and Mathematics, DREC Department, Bachelors second-year
student}}

\end{rSection}

%----------------------------------------------------------------------------------------
%	SKILLS
%----------------------------------------------------------------------------------------
\begin{rSection}{Skills}

\begin{tabular}{ @{} >{\bfseries}l @{\hspace{6ex}} l }
Languages: \ & C/C++, Python, Bash \\

Software \& Tools: & CMake, Make, Git \\

Other Skills: & CPU micro-architecture knowledge, English (B1), \LaTeX , Russian (C2)
\end{tabular}

\end{rSection}

%--------------------------------------------------------------------------------
%    PROJECTS
%--------------------------------------------------------------------------------
\begin{rSection}{Projects (All available on Github)}

{\bf Second-year ILAB projects} \hfill {\em Sep 2021 - May 2022}
\\ {Repository: $\href {https://github.com/VasiliyMatr/ILAB_2ndYEAR}
    {VasiliyMatr/ILAB\_2ndYEAR}$}
\\
\\ There are good projects written in C++ and built with CMake in this repository.

{\bf Bachelors first-term projects} \hfill {\em Sep 2020 - Jan 2021}
\\ Repository: $\href {https://github.com/VasiliyMatr/MIPT_PROG_1stTERM}
    {VasiliyMatr/MIPT\_PROG\_1stTERM}$
\\- Assembly compiler, which produces executables. There is also a CPU simulator to run
    such executables:
    $\href {https://github.com/VasiliyMatr/MIPT_PROG_1stTERM/tree/master/T04CPU}{T04CPU}$
\\- Toy programming language compiler, which produces assembly outputs that are used in
    T04CPU:
    $\href {https://github.com/VasiliyMatr/MIPT_PROG_1stTERM/tree/master/T08LANG}{T08LANG}$


{\bf Bachelors second-term projects} \hfill {\em Feb 2021 - May 2021}
\\ Repository: $\href {https://github.com/VasiliyMatr/MIPT_PROG_2ndTERM}
    {VasiliyMatr/MIPT\_PROG\_2ndTERM}$
\\- Optimizations with intrinsics:
    $\href {https://github.com/VasiliyMatr/MIPT_PROG_2ndTERM/tree/master/T05MANDEL}{T05MANDEL}$,
    $\href {https://github.com/VasiliyMatr/MIPT_PROG_2ndTERM/tree/master/T06HASH}{T06HASH}$
\\- T04CPU executables to x86 executables binary translator:
    $\href {https://github.com/VasiliyMatr/MIPT_PROG_2ndTERM/tree/master/T07JIT}{T07JIT}$
\\- There is also some assembly code in a few projects.
\end{rSection}

%----------------------------------------------------------------------------------------
%	WORK EXPERIENCE
%----------------------------------------------------------------------------------------
\begin{rSection}{Work Experience}

{\bf Intel corporation, IPP Data Compression Domain} \hfill {\em Jul 2021 - Aug 2021}
\\{\textit{ Summer intern}}
\\
\\ My manager and mentor highly valued my work and sent good
    $\href {https://github.com/VasiliyMatr/CV/tree/master/PDF/ISI2021_feedback.pdf}{feedback}$

\end{rSection}

%----------------------------------------------------------------------------------------
%	WORK EXPERIENCE
%----------------------------------------------------------------------------------------
\begin{rSection}{Finished additional courses}

{\bf Introduction to Industrial Programming and Data Structures, Mail.ru} \hfill
    {\em Sep 2020 - Jan 2021}
\\ Lecturer - I. Dedinsky.

{\bf Operating System and CPU Simulation, HiSilicon} \hfill {\em Sep 2021 - May 2022}
\\ Lecturer - I. Petushkov.

{\bf C++ Basic Course, ILAB} \hfill {\em Sep 2021 - May 2022}
\\ Lecturer - K. Vladimirov.

{\bf CPU micro-architecture, ILAB} \hfill {\em Sep 2021 - May 2022}
\\ Lecturer - K. Korolev.

\vspace{0.3cm}

{\bf STM32 Microcontrollers Introduction, MIPT DREC} \hfill {\em Sep 2020 - Jan 2021}

{\bf FPGA and Verilog Introduction, MIPT DREC} \hfill {\em Feb 2021 - May 2021}

\end{rSection}


\end{document}